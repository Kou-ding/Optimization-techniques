% Document class and two-column conversion
\documentclass{report}
% dimensions of paper and relative text positioning
\usepackage[a4paper,top=2cm,bottom=2cm,left=2cm,right=2cm]{geometry}
% package for including URLs
\usepackage{url}
% Required for including images
\usepackage{graphicx}
\usepackage{float} % Required for specifying the exact location of a figure

% enable writing in greek
\usepackage[greek,english]{babel}
\usepackage[utf8]{inputenc}
%\usepackage[LGR]{fontenc}

\setlength{\parindent}{0pt} % Removes all indentation from paragraphs

% Start of the document
\begin{document}

% Set the language to greek
\selectlanguage{greek}

% Title page
\title{\Huge \bfseries Τεχνικές Βελτιστοποίησης \\ Δεύτερο Παραδοτέο} %\Huge and \bfseries are used to make the title bigger and bold
\author{Παπαδάκης Κωνσταντίνος Φώτιος\vspace{0.5cm} \\  ΑΕΜ:10371} % \vspace{0.5cm} is used to add some vertical space between the author and the AEM
\date{\today}
% prints the title, author and date on a separate page
\maketitle

% Table of contents page
\tableofcontents

% General introduction
\chapter{Εισαγωγή}
\section{Γενικά}
Η συνάρτηση πάνω στην οποία θα δουλέψουμε είναι η εξής:
$$f(x) = x^5 e^{-x^2-y^2}$$

'Επειτα σχεδιάζουμε ένα ακόμα \selectlanguage{english}Matlab script
\selectlanguage{greek}το οποίο παράγει τις πρώτες και δεύτερες μερικές παραγώγους της 
συνάρτησης $f(x,y)$:

\textbf{Πρώτη Μερική Παράγωγος}
\begin{itemize}
    \item Ως προς το $x$:
    $$\frac{\partial{z}}{\partial{x}} = 5x^4e^{-x^2-y^2} - 2x^6e^{-x^2-y^2}$$
    $$\frac{\partial{z}}{\partial{x}} = x^4(5-2x^2)e^{-x^2-y^2}$$
    \item Ως προς το $y$:
    $$\frac{\partial{z}}{\partial{y}} = -2x^{5}ye^{-x^2-y^2}$$

\end{itemize}

\textbf{Δεύτερη Μερική Παράγωγος}
\begin{itemize}
    \item Ως προς το $x$:
    $$\frac{\partial^2{z}}{\partial{x^2}} = 20x^{3}e^{-x^2-y^2}-22x^{5}e^{-x^2-y^2}+4x^{7}e^{-x^2-y^2}$$
    $$\frac{\partial^2{z}}{\partial{x^2}} = (20x^3 -22x^5 +4x^7)e^{-x^2-y^2}$$
    $$\frac{\partial^2{z}}{\partial{x^2}} = 2x^3(10-11x^2+2x^4)e^{-x^2-y^2}$$
    $$\frac{\partial^2{z}}{\partial{x^2}} = 2x^3(2x^2-1)(x^2-10)e^{-x^2-y^2}$$
    \item Ως προς το $y$:
    $$\frac{\partial^2{z}}{\partial{y^2}} = 4x^{5}y^{2}e^{-x^2-y^2} - 2x^{5}e^{-x^2-y^2}$$
    $$\frac{\partial^2{z}}{\partial{y^2}} = 2x^{5}(2y^2-1)e^{-x^2-y^2}$$
\end{itemize}

% Visualization of the function
\chapter{Θέμα 1}
\section{Εκφώνηση}
Σχεδιάστε την $f$ για να πάρετε μια γενική εικόνα της μορφής της.
\section{Λύση}
Χρησιμοποιώντας ένα απλό \selectlanguage{english}visualization script
\selectlanguage{greek}σε
\selectlanguage{english}Matlab
\selectlanguage{greek}εξάγουμε την παρακάτω εικόνα:
\begin{figure}[H]
    \centering
    \includegraphics[width=0.5\textwidth]{media/visualization.png}
    \caption{Οπτικοποίηση της συνάρτησης $f(x,y)$}
\end{figure}

% Steepest Descent
\chapter{Θέμα 2}
\section{Εκφώνηση}
Ελαχιστοποιήστε την $f$ με την μέθοδο Μέγιστης Καθόδου, χρησιμοποιώντας ως αρχικά σημεία 
$(x, y)$ τα $i)$ $(0,0)$, $ii)$ $(-1,1)$, και $iii)$ $(1,-1)$. Το βήμα γ θα επιλεγεί: 
\begin{itemize}
    \item α) σταθερό (της επιλογής σας), 
    \item β) τέτοιο ώστε να ελαχιστοποιεί την $f(x + γd)$ και 
    \item γ) βάσει του κανόνα \selectlanguage{english}Armijo.
\end{itemize}
\selectlanguage{greek}
Σχολιάστε τις διαφορές στα αποτελέσματα, σε περίπτωση που προκαλούνται, λόγω της επιλογής του 
σημείου έναρξης $(x, y)$ του αλγορίθμου, καθώς επίσης και λόγω της επιλογής του βήματος γ.
δηγούμαστε πάντα σε σωστό αποτέλεσμα; Αν όχι, τι πιστεύετε ότι φταίει;

Σημείωση: Στην περίπτωση σταθερού βήματος δε χρειάζεται μαθηματική ανάλυση για τη συνθήκη σύγκλισης. 
Με βάση τη θεωρία εφαρμόστε τις τιμές απευθείας στο \selectlanguage{english}Matlab.
\selectlanguage{greek}
\section{Λύση}
Στη μέθοδο της Μέγιστης Κοθόδου υπολογίζουμε την μερική παράγωγο της συνάρτησης $f(x,y)$
έτσι ώστε να μεταβούμε προς την κατεύθυνση όπου η συνάρτηση φθίνει και παράλληλα πλησιάζει
πιο κοντά στο ελάχιστο της. Η εκτέλεση του κώδικα σταματάει όταν φτάσουμε τον μέγιστο αριθμό
επαναλήψεων που ορίσαμε αυθαίρετα ή όταν η απόλυτη τιμή της παραγώγου της συνάρτησης γίνει 
μικρότερη από μια σταθερά ανοχής που μας επιτρέπει να πλησιάσουμε αρκετά στο σημείο ελάχιστου
έτσι ώστε να τελειώνει γρηγορότερα ο αλγόριθμός μας.

Τα σημεία εκκίνησης που μας δώθηκαν είναι τα εξής:
\begin{itemize}
    \item [0,0]: Είναι ένα σημείο στο οποίο η παράγωγος είναι εξαρχής μηδέν και άρα ο αλγόριθμος
    σταματάει αμέσως, όσο μικρή σταθερά ανοχής και να βάλω.
    \item [-1,1]: 
    \item [1,-1]:
\end{itemize}
Το βήμα με το οποίο μετακινούμαστε στον χώρο το ορίζουμε με έναν από ο τους εξής τρόπους:
\begin{itemize}
    \item Σταθερό βήμα\\
    \textbf{Παρατηρήσεις:} Έχοντας ορίσει βήμα 0,001 και ανοχή 0,001 παρατηρούμε ότι ο 
    αλγόριθμος συγκλίνει μετά 5214 επαναλήψεις, στην καλύτερη περίπτωση που ξεκινώ από το σημείο (-1,1)
    η οποία είναι η καλύτερη περίπτωση διότι  από 942 επαναλήψεις έτσι ώστε η παράγωγος της να γίνει μικρότερη από το 0,00001
    \item Βήμα που ελαχιστοποιεί την $f(x + γd)$\\
    \textbf{Παρατηρήσεις:} 
    \item Βήμα βάσει του κανόνα \selectlanguage{english}Armijo\selectlanguage{greek}\\
    \textbf{Παρατηρήσεις:} 
\end{itemize}




% Newton
\chapter{Θέμα 3}
\section{Εκφώνηση}
Επαναλάβετε τα ερωτήματα του Θέματος 2 χρησιμοποιώντας την μέθοδο 
\selectlanguage{english}Newton.
\selectlanguage{greek}
\section{Λύση}


% Levenberg-Marquardt
\chapter{Θέμα 4}
\section{Εκφώνηση}
Επαναλάβετε τα ερωτήματα του Θέματος 2 χρησιμοποιώντας την μέθοδο 
\selectlanguage{english}Levenberg-Marquardt.
\selectlanguage{greek} 
\section{Λύση}


% Comparison
\chapter{Ανάλυση Αποτελεσμάτων}
\section{Σύγκριση Μεθόδων}

\section{Σχόλια}


% Bibliography
\nocite{*} % Include all references in the bibliography, even if they are not cited in the report
\bibliographystyle{plain}
\bibliography{references/references} % We have to include the references somewhere in the report for them to show here if we don't use (\nocite{*})
\addcontentsline{toc}{chapter}{\bibname} % Add the bibliography to the table of contents

% End of the document
\end{document}