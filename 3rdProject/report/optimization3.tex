% Document class and two-column conversion
\documentclass{report}
% dimensions of paper and relative text positioning
\usepackage[a4paper,top=2cm,bottom=2cm,left=2cm,right=2cm]{geometry}
% math symbols
\usepackage{amsmath}
\usepackage{amssymb}
% package for including URLs
\usepackage{url}
% Required for including images
\usepackage{graphicx}
\usepackage{float} % Required for specifying the exact location of a figure

% enable writing in greek
\usepackage[greek,english]{babel}
\usepackage[utf8]{inputenc}
%\usepackage[LGR]{fontenc}

\setlength{\parindent}{0pt} % Removes all indentation from paragraphs

% Start of the document
\begin{document}

% Set the language to greek
\selectlanguage{greek}

% Title page
\title{\Huge \bfseries Τεχνικές Βελτιστοποίησης \\ Τρίτο Παραδοτέο} %\Huge and \bfseries are used to make the title bigger and bold
\author{Παπαδάκης Κωνσταντίνος Φώτιος\vspace{0.5cm} \\  ΑΕΜ:10371} % \vspace{0.5cm} is used to add some vertical space between the author and the AEM
\date{\today}
% prints the title, author and date on a separate page
\maketitle

% Table of contents page
\tableofcontents

% General introduction
\chapter{Εισαγωγή}
\section{Ανάλυση Συνάρτησης $f(x)$}
Η συνάρτηση που θα εξετάσουμε στο τρίτο παραδοτέο είναι η εξής:
$$ f: \mathbb{R}^2 \to \mathbb{R},\; f(x) = \frac{1}{3} x_1^2 + 3x_2^2,\; x = [x_1 x_2]^T$$
Η μορφή της στον τρισδιάστατο χώρο είναι η εξής:
\begin{figure}[H]
    \centering
    \includegraphics[width=0.5\textwidth]{media/visual.png}
    \caption{Οπτικοποίηση συνάρτησης $f(x,y)$}
\end{figure} 
Παρατηρούμε ότι η συνάρτηση σημειώνει ολικό ελάχιστο στο σημείο $(0,0)$.

% Steepest Descent: Various constant steps
\chapter{Θέμα 1}
\section{Εκφώνηση}
Να χρησιμοποιηθεί η Μέθοδος Μέγιστης Καθόδου (προηγούμενη εργασία) με ακρίβεια $\epsilon =
0.001$ και βήμα:
\begin{itemize}
    \item i) $\gamma_k$ = 0.1
    \item ii) $\gamma_k$ = 0.3
    \item iii) $\gamma_k$ = 3
    \item iv) $\gamma_k$ = 5
\end{itemize}
και οποιοδήποτε αρχικό σημείο εκκίνησης διαφορετικό του $(0,0)$. Τι παρατηρείτε
\selectlanguage{english};\selectlanguage{greek} Να αποδειχθούν τα αποτελέσματα αυτά 
με μαθηματική αυστηρότητα.
\section{Λύση}
Επέλεξα το σημείο εκκίνησης $(-10,-10)$ ως ένα σημείο αρκετά απομαρκυσμένο από το ολικό ελάχιστο, 
κρίνοντας πάντα από την οπτικοποίηση της συνάρτησης στο \selectlanguage{english}Visualization.m script
\selectlanguage{greek}. Έτσι θα μπορέσουμε να παρακολουθήσουμε καλύτερα την σύγκλιση του αλγορίθμου.
Το διάγραμμα που προκύπτει από την εκτέλεση του αλγορίθμου Μέγιστης Καθόδου με διάφορα σταθερά βήματα 
απέκρυβε πληροφορία οπότε αποφάσισα να το σπάσω σε δύο επιμέρους διαγράμματα το καθένα με βήματα που
παρουσιάζουν παρόμοια συμπεριφορά:
\begin{itemize}
    \item Διάγραμμα που συγκλίνει στο ελάχιστο:
    \begin{figure}[H]
        \centering
        \includegraphics[width=0.7\textwidth]{media/thema1-1.png}
        \caption{Διάγραμμα Μέγιστης Καθόδου με σταθερά βήματα: 0.1, 0.3}
    \end{figure}
    \item Διάγραμμα που κάθε επανάληψη απομακρύνεται ολοένα και περισσότερο από το ελάχιστο:
    \begin{figure}[H]
        \centering
        \includegraphics[width=0.7\textwidth]{media/thema1-2.png}
        \caption{Διάγραμμα Μέγιστης Καθόδου με σταθερά βήματα: 3, 5}
    \end{figure}
    Σε κάθε επανάληψη η απομάκρυνση που σημειώνεται είναι αρκετά μεγαλύτερη από αυτή της
    προηγούμενης επανάληψης έτσι ώστε το η γραφική παράσταση της $f(x,y)$ μέχρι εκείνο το 
    σημείο να φαντάζει σαν να ταυτίζεται με τον άξονα $x'x$. 
    \begin{figure}[H]
        \centering
        \includegraphics[width=0.8\textwidth]{media/thema1-2-2.png}
        \caption{Διάγραμμα Μέγιστης Καθόδου με σταθερά βήματα: 3, 5}
    \end{figure}
\end{itemize}

\subsection{Μαθηματική Απόδειξη}
Αναλυτικά έχουμε:
$$f(x) = \frac{1}{3} x_1^2 + 3x_2^2$$
\begin{center}
    και
\end{center}
$$\frac{\partial f}{\partial x_1} = \frac{2}{3}x_1$$
$$\frac{\partial f}{\partial x_2} = 6x_2$$
Η εξίσωση που μας δίνει την σύγκλιση είναι:
$$x_{k+1} = x_k - \gamma_k \nabla f(x_k)$$
Χρησιμοποιώντας πίνακες, έχουμε:
$$x_{k+1} = \begin{bmatrix} x_{1k} \\ x_{2k} \end{bmatrix} - \gamma_k \begin{bmatrix} \frac{2}{3}x_{1k} \\ 6x_{2k} \end{bmatrix}$$
$$x_{k+1} = \begin{bmatrix} x_{1k} - \frac{2}{3}\gamma_k x_{1k} \\ x_{2k} - 6\gamma_k x_{2k} \end{bmatrix}$$
$$x_{k+1} = \begin{bmatrix} (1 - \frac{2}{3}\gamma_k)x_{1k} \\ (1 - 6\gamma_k)x_{2k} \end{bmatrix}$$
$$x_{k+1} = \begin{bmatrix} 1 - \frac{2}{3}\gamma_k & 0 \\ 0 & 1 - 6\gamma_k \end{bmatrix} \begin{bmatrix} x_{1k} \\ x_{2k} \end{bmatrix}$$
$$x_{k+1} = M \cdot x_k$$

Για να πούμε ότι για να συγκλίνει ο αλγόριθμος πρέπει η απόλυτη τιμή των ιδιοτιμών του πίνακα M: 
$$M = \begin{bmatrix} 1 - \frac{2}{3}\gamma_k & 0 \\ 0 & 1 - 6\gamma_k \end{bmatrix}$$
να είναι μικρότερες του 1. Οπότε:
$$|\lambda_1|= 1 - \frac{2}{3}\gamma_k < 1$$
$$|\lambda_2| = 1 - 6\gamma_k < 1$$

Οι οποίες γίνονται:
$$0 < \gamma_k < 3$$
$$0 < \gamma_k < \frac{1}{3}$$
Έτσι συνεπάγουμε ότι σωστά στο διάγραμμα παρατηρούμε σύγκλιση για τα βήματα 0.1 και 0.3 και 
απομάκρυνση για τα βήματα 3 και 5.


\subsection{Εισαγωγή Περιορισμών}
Θεωρήστε τώρα τους περιορισμούς:
$$-10 <= x_1 <= 5$$  
\begin{center}
    και
\end{center}
$$-8 <= x_2 <= 12$$

% Steepest Descent: Projection [1]
\chapter{Θέμα 2}
\section{Εκφώνηση}
Να χρησιμοποιηθεί η Μέθοδος Μέγιστης Καθόδου με Προβολή, με $s_k = 5$, $\gamma_k = 0.5$,
σημείο εκκίνησης το $(5, -5)$ και ακρίβεια $\epsilon = 0.01$. Τι παρατηρείτε σε σχέση με το 
Θέμα 1\selectlanguage{english};\selectlanguage{greek}

\section{Λύση}

Η γραφική αναπαράσταση των αποτελεσμάτων παραθέτεται παρακάτω:
\begin{figure}[H]
    \centering
    \includegraphics[width=0.7\textwidth]{media/thema2.png}
    \caption{Διάγραμμα Μέγιστης Καθόδου με προβολή $s_k = 5$, $\gamma_k = 0.5$}
\end{figure}
Το ακριβές σημείο που σταμάτησε ο αλγόριθμος ήταν $f(0.0, 0.0) = 0.000$ έπειτα από 
$χχχχχ$ επαναλήψεις.

% Steepest Descent: Projection [2]
\chapter{Θέμα 3}
\section{Εκφώνηση}
Να χρησιμοποιηθεί η Μέθοδος Μέγιστης Καθόδου με Προβολή, με $s_k = 15$, $\gamma_k = 0.1$,
σημείο εκκίνησης το $(-5,10)$ και ακρίβεια $\epsilon = 0.01$. Τι παρατηρείτε σε σχέση με τα Θέματα 1 
και 2\selectlanguage{english};\selectlanguage{greek} Προτείνετε έναν απλό πρακτικό τρόπο ώστε η 
μέθοδος να συγκλίνει στο ελάχιστο.

\section{Λύση}
Η γραφική αναπαράσταση των αποτελεσμάτων παραθέτεται παρακάτω:
\begin{figure}[H]
    \centering
    \includegraphics[width=0.7\textwidth]{media/thema3.png}
    \caption{Διάγραμμα Μέγιστης Καθόδου με προβολή $s_k = 15$, $\gamma_k = 0.1$}
\end{figure}
Το ακριβές σημείο που σταμάτησε ο αλγόριθμος ήταν $f(0.0, 0.0) = 0.000$ έπειτα από
$χχχχχ$ επαναλήψεις.


% Steepest Descent: Projection [3]
\chapter{Θέμα 4}
\section{Εκφώνηση}
Να χρησιμοποιηθεί η Μέθοδος Μέγιστης Καθόδου με Προβολή, με $s_k = 0.1$, $\gamma_k = 0.2$,
σημείο εκκίνησης το $(8, -10)$ και ακρίβεια $\epsilon = 0.01$. Σε αυτή την περίπτωση, έχουμε εκ των
προτέρων κάποια πληροφορία σχετικά με την σύγκλιση του αλγορίθμου\selectlanguage{english};
\selectlanguage{greek}Να γίνει η εκτέλεση του αλγορίθμου. Τι παρατηρείτε
\selectlanguage{english};\selectlanguage{greek}

\section{Λύση}

Η γραφική αναπαράσταση των αποτελεσμάτων παραθέτεται παρακάτω:
\begin{figure}[H]
    \centering
    \includegraphics[width=0.7\textwidth]{media/thema4.png}
    \caption{Διάγραμμα Μέγιστης Καθόδου με προβολή $s_k = 0.1$, $\gamma_k = 0.2$} 
\end{figure}
Το ακριβές σημείο που σταμάτησε ο αλγόριθμος ήταν $f(0.0, 0.0) = 0.000$ έπειτα από
$χχχχχ$ επαναλήψεις.

% Comparison
\chapter{Ανάλυση αποτελεσμάτων}

\section{Σχόλια}


% Bibliography
\nocite{*} % Include all references in the bibliography, even if they are not cited in the report
\bibliographystyle{plain}
\bibliography{references/references} % We have to include the references somewhere in the report for them to show here if we don't use (\nocite{*})
\addcontentsline{toc}{chapter}{\bibname} % Add the bibliography to the table of contents

% End of the document
\end{document}