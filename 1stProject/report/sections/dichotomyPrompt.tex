Υλοποιήστε στο \selectlanguage{english}Matlab \selectlanguage{greek} τη μέθοδο της Διχοτόμου και εφαρμόστε τη στις 
συναρτήσεις $f_1(x)$, $f_2(x)$, $f_3(x)$.
\begin{itemize}
    \item Κρατώντας σταθερό το τελικό εύρος αναζήτησης $l = 0.01$ μελετήστε τη μεταβολή των
υπολογισμών της αντικειμενικής συνάρτησης $f_i(x)$, $i = 1,2,3$ (δηλαδή τον συνολικό αριθμό που
χρειάστηκε να υπολογιστεί η $f(x)$, για τις δεδομένες τιμές των $l$ και $\epsilon$, μέχρι να τερματίσει ο
αλγόριθμος), καθώς μεταβάλλουμε τη σταθερά $\epsilon > 0$ (απόσταση από τη διχοτόμο).
Δημιουργήστε τις αντίστοιχες γραφικές παραστάσεις από τις τιμές που προκύπτουν για τις τρεις
συναρτήσεις.
    \item Κρατώντας σταθερό το $\epsilon = 0.001$ μελετήστε τη μεταβολή των υπολογισμών της $f_i(x)$, 
$i=1,2,3$, καθώς μεταβάλλουμε το $l$. Δημιουργήστε τις αντίστοιχες γραφικές παραστάσεις από τις
τιμές που προκύπτουν για τις τρεις συναρτήσεις.
    \item Επιπλέον, σε τρία διαγράμματα, ένα για κάθε συνάρτηση, σχεδιάστε τις γραφικές παραστάσεις
των άκρων του διαστήματος $[a_k, b_k]$ συναρτήσει του δείκτη επαναλήψεων $k$, δηλαδή $(k, a_k)$ και
$(k, b_k)$, για διάφορες τιμές του τελικού εύρους αναζήτησης $l$.
\end{itemize}