Προτού μπούμε αναλυτικά στην λειτουργία κάθε αλγορίθμου χρειάζεται να κάνουμε κάποιες γενικές
παρατηρήσεις για να αποφύγουμε την επανάληψη αργότερα. Και οι τρεις συναρτήσεις πάνω στις οποίες
θα δουλέψουμε είναι κυρτές. 

Οι δεύτερες παράγωγοι έχουν ως εξής:

$\frac{d^2f_1}{dx^2} = \frac{2}{x + 3} - \frac{x}{(x + 3)^2} + 2$

$\frac{d^2f_2}{dx^2} = 4e^{-2x} + 2$

$\frac{d^2f_3}{dx^2} = 2cos(x) + 6x^2e^{x} - sin(x)(x - 1) + e^{x}(x^3 - 1) + 6xe^{x}$

Στο διαστημα $[-1,3]$ όπου και εφαρμόζουμε τους αλγορίθμους μας οι δεύτερες παράγωγοι των $f_1$, $f_2$ είναι
θετικές και συμπερασματικά οι αρχικές συναρτήσεις τους είναι κυρτές. Η $f_3$ είναι κοίλη στο διάστημα
$[-1,-0.236347]$ και κυρτή στο διάστημα $[-0.236347,3]$. Όμως παρατηρώντας την πρώτη παράγωγο της
$f_3$ στο διάστημα $[-1,3]$ βλέπουμε ότι αλλάζει μόνο μια φορά πρόσημο στο [0.52009,0].

$\frac{df_3}{dx} = e^x * (x^3 - 1) + (x - 1) * sin(x)$

Οπότε η $f_3$ είναι σχεδόν κυρτή στο διάστημα $[-1,3]$. Συνεπώς μας επιτρέπεται να χρησιμοποιήσουμε τους
παρακάτω αλγορίθμους. Ένας εναλλακτικός τρόπος εργασίας πάνω στην τρίτη συνάρτηση θα ήταν να σπάσουμε
τα δύο υπο-διαστήματα στα οποία η $f_3$ είναι κυρτή και κοίλη και να εφαρμόσουμε τον αλγόριθμο σε κάθε
ένα από αυτά ξεχωριστά. Το μικρότερο από τα δύο ελάχιστα θα είναι και το ολικό ελάχιστο της $f_3$ στο 
διάστημα $[-1,3]$.
