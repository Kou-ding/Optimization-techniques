Προτού μπούμε αναλυτικά στην λειτουργία κάθε αλγορίθμου χρειάζεται να κάνουμε κάποιες γενικές
παρατηρήσεις για να αποφύγουμε την επανάληψη αργότερα. Και οι τρεις συναρτήσεις πάνω στις οποίες
θα δουλέψουμε είναι κυρτές. 

Οι δεύτερες παράγωγοι έχουν ως εξής:

$\frac{d^2f_1}{dx^2} = \frac{2}{x + 3} - \frac{x}{(x + 3)^2} + 2$

$\frac{d^2f_2}{dx^2} = 4e^{-2x} + 2$

$\frac{d^2f_3}{dx^2} = 2cos(x) + 6x^2e^{x} - sin(x)(x - 1) + e^{x}(x^3 - 1) + 6xe^{x}$

Στο διαστημα $[-1,3]$ όπου και εφαρμόζουμε τους αλγορίθμους μας οι δεύτερες παράγωγοι των $f_1$, $f_2$ είναι
θετικές και συμπερασματικά οι αρχικές συναρτήσεις τους είναι κυρτές. Η $f_3$ είναι κοίλη στο διάστημα
$[-1,-0.236347]$ και κυρτή στο διάστημα $[-0.236347,3]$. 'Ετσι δεν αναμένουμε να συγκλίνει με τη χρήση 
των αλγοριθμικών μας μεθόδων.

