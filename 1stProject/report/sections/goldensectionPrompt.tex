\subsection Θέμα 2
Υλοποιήστε στο Matlab τη μέθοδο του Χρυσού Τομέα και εφαρμόστε τη στις συναρτήσεις
𝑓ଵ(𝑥), 𝑓ଶ(𝑥), 𝑓ଷ(𝑥).

Μελετήστε τη μεταβολή των υπολογισμών της αντικειμενικής συνάρτησης 𝑓௜(𝑥), 𝑖 =
1,2,3 (δηλαδή τον συνολικό αριθμό που πρέπει να υπολογιστεί η 𝑓௜(𝑥) μέχρι να τερματίσει ο
αλγόριθμος), καθώς μεταβάλλουμε το τελικό εύρος αναζήτησης 𝑙. Δημιουργήστε τις αντίστοιχες
γραφικές παραστάσεις από τις τιμές που προκύπτουν για τις τρεις συναρτήσεις.

Επιπλέον, σε τρία διαγράμματα, ένα για κάθε συνάρτηση, σχεδιάστε τις γραφικές παραστάσεις
των άκρων του διαστήματος [𝑎௞, 𝑏௞] συναρτήσει του δείκτη επαναλήψεων 𝑘, δηλαδή
(𝑘, 𝑎௞) και (𝑘, 𝑏௞), για διάφορες τιμές του τελικού εύρους αναζήτησης 𝑙.