\subsection Κατανόηση Αλγορίθμου

Χωρίζουμε το αρχικό μας χωρίο σε τρία ανισομερή υπό-διαστήματα χρησιμοποιώντας τα σημεία $x1$ και $x2$.
Τα σημεία αυτά βρίσκονται πολύ κοντά το ένα με το άλλο, και χάρη στην υπόθεσή μας ότι η συνάρτηση που
εξετάζουμε είναι αυστηρά σχεδόν κυρτή, καταλαβαίνουμε ότι στην περίπτωση που τραβούσαμε μια εφαπτομενική 
με τα δύο σημείακαι στις δύο περιπτώσεις η συνάρτηση θα βρισκεται πάνω από την ευθεία αυτή. Έτσι 
συμπεραίνουμε ότι δεν μπορεί να υπάρξει ελάχιστο το οποίο να βρίσκεται πιο δεξιά από το $x2$, στην 
περίπτωση που $f(x_1) < f(x_2)$, και αντίστοιχα δεν μπορεί να υπάρξει ελάχιστο το οποίο να βρίσκεται πιο
αριστερά από το $x1$, στην περίπτωση που $f(x_1) >= f(x_2)$. Η μείωση του χωρίου αναζήτησης συνεχίζει
μέχρι να φτάσουμε σε χωρίο μικρότερο από μια σταθερά $l$. Σημαντικό να σημειωθεί ότι η ακρίβεια του 
αλγορίθμου μας δεν μπορεί να είναι μικρότερη από $2*\epsilon$. Αυτό συμβαίνει γιατί στο υποθετικό σενάριο
όπου παίρνουμε $l$ πολύ μικρό, γεγονός το οποίο αποτρέπει τον αλγόριθμο από το να σταματήσει, όταν το
χωρίο αναζήτησης γίνει ίσο του $2*\epsilon$ τότε δεν μπορούμε να κάνουμε καμία περεταίρω ουσιώδη μείωση
του χωρίου αναζήτησης καθώς τα $x1=\frac{a+b}{2}-\epsilon$ και $x2=\frac{a+b}{2}+\epsilon$ θα συνεχίζουν 
να ορίζουν το ίδιο χωρίο αναζήτησης.


\subsection Βήματα Αλγορίθμου

\begin{enumerate}
    \item Ελέγχουμε αν $b - a < l$. Αν ισχύει, σταματάμε την αναζήτηση.
    \item Στην περίπτωση που δεν ισχύει, θέτουμε $x1=\frac{a+b}{2}-\epsilon$ και
    $x2=\frac{a+b}{2}+\epsilon$.
    \item Αν $f(x_1) < f(x_2)$, θέτουμε $a_{k+1}=a_k$ και $b_{k+1}=x_{2k}$.
    \item Αλλιώς θέτουμε $a_(k+1)=x_(1k)$ και $b_(k+1)=b_k$.
    \item Μεταβαίνουμε στην επόμενη επανάληψη του αλγορίθμου $k=k+1$ και σταματάμε όταν
    $b_k - a_k < l$.
\end{enumerate}