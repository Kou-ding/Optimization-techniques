Κατανόηση Αλγορίθμου

Χωρίζουμε το αρχικό μας χωρίο σε τρία ανισομερή υπό-διαστήματα χρησιμοποιώντας τα σημεία $x1$ και $x2$.
To determine the correct subinterval to commence our new search we use the Θεώρημα 5.1.1 του βιβλίου.
Σύμφωνα με αυτό, αν μια συνάρτηση $f$ είναι \textbf{αυστηρά σχεδόν κυρτή} σε ένα διάστημα $[a,b]$
όπου $x_1,x_2 \epsilon[a,b]$ και $a < x_1 < x_2 < b$, τότε αν $f(x_1)<f(x_2)$, $f(x)>=f(x_1)$ για κάθε
$x \epsilon [x_2,b]$. Ομοίως, αν $f(x_1)>=f(x_2)$, $f(x)>=f(x_2)$ για κάθε $x \epsilon [a,x_1]$.

Βήματα Αλγορίθμου:
\begin{enumerate}
    \item Ελέγχουμε αν $b - a < l$. Αν ισχύει, σταματάμε την αναζήτηση.
    \item Στην περίπτωση που δεν ισχύει, θέτουμε $x1=\frac{a+b}{2}-\epsilon$ και
    $x2=\frac{a+b}{2}+\epsilon$.
    \item Αν $f(x_1) < f(x_2)$, θέτουμε $a_{k+1}=a_k$ και $b_{k+1}=x_{2k}$.
    \item Αλλιώς θέτουμε $a_(k+1)=x_(1k)$ και $b_(k+1)=b_k$.
    \item Μεταβαίνουμε στην επόμενη επανάληψη του αλγορίθμου $k=k+1$ και σταματάμε όταν
    $b_k - a_k < l$.
\end{enumerate}